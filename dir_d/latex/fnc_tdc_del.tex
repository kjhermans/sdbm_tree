
\newpage
\subsection{tdc\_del}
\subsubsection{Declaration} Function prototype:

\begin{verbatim}
extern
int tdc_del
  (tdc_t* tdc, tdt_t* key, tdt_t* value, unsigned flags);
\end{verbatim}

\subsubsection{Description}


 ingroup btree

 Removes the object at the cursor.

 par NOTE
 This function invalidates the cursor after successful execution,
 because the structure of the tree cannot be relied on anymore.
 (ie You can call this function once).

 par NOTE
 If you call this function after a tdc\_nxt(), as is common in loops,
 you may not remove the element that you think you're removing.

 param tdc    Pointer to an initialized tdc\_t structure.
 param key    Optional pointer to an initialized dbt\_t structure.
               When non-NULL, is used to return the key at the cursor.
 param value  Optional pointer to an initialized dbt\_t structure.
               When non-NULL, is used to return the value at the cursor.
 param flags  Flags; these only pertain to the 'get' portion of this
               function and are ignored otherwise.
               if TDFLG\_ALLOCTDT is given, and the key or value
               size is NULL, room in them will be allocated using the
               allocator function.

 returns Zero on success, TDERR\_NOTFOUND if the cursor is not positioned
          somewhere viable, or TDERR\_INVAL when the System Change Number
          has invalidated the cursor (cursor is stale).
          Or any other errors of any underlying functions.
 

\subsubsection{Parameters}
\subsubsection{Returns}
\subsubsection{Called by}
