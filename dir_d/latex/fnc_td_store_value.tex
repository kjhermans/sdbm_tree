
\newpage
\subsection{td\_store\_value}
\subsubsection{Declaration} Function prototype:

\begin{verbatim}
extern
int td_store_value
  (
    td_t* td,
    const tdt_t* value,
    unsigned valuecount,
    unsigned refcount,
    unsigned* off,
    unsigned flags
  );
\end{verbatim}

\subsubsection{Description}


 Stores an empty value, that is: the zero length string.
 This special case is optimized for because the claim() function
 isn't allowed to return a buffer less than the size of a header.
/
static
int td\_store\_empty\_value
  (td\_t* td, unsigned refcount, unsigned* off, unsigned flags)
{
  unsigned chunkoff, given = sizeof(struct chunkhead);
  struct chunkhead h = {
    .next = 0,
    .size = given,
    .checksum = 0,
    .refcount = refcount
  };
  (void)flags;

  if (flags & TDFLG\_CHECKSUM) {
    td\_checksum\_create(0, 0, &(h.checksum));
  }
  CHECK(td\_claim(td, 0, &chunkoff, &given));
  if (given != sizeof(struct chunkhead)) {
    return TDERR\_STRUCT;
  }
  CHECK(td\_write\_chunkhead(td, chunkoff, &h));
  ++(td->header.nchunks);
  *off = chunkoff;
  return 0;
}

struct tdvbuf
{
  tdt\_t             value;
  unsigned          used;
};

struct tdvvec
{
  struct tdvbuf*    list;
  unsigned          offset;
  unsigned          count;
};

/**
 Optimization: when the given space is smaller than or equal to
 the amount of data in my current value, then the current value
 will not require a copy to be used to store (part of its) data.
/
static
int td\_store\_chunk\_nocopy
  (
    td\_t*           td,
    unsigned        chunkoff,
    unsigned        chunkdatasize,
    unsigned        nextchunk,
    struct tdvvec*  vec,
    unsigned        refcount,
    unsigned        flags
  )
{
  struct tdvbuf* curval = &(vec->list[ vec->offset ]);

  CHECK2(
    td\_store\_valuechunk(
      td,
      chunkoff,
      nextchunk,
      curval->value.data + curval->value.size - (curval->used + chunkdatasize),
      chunkdatasize,
      refcount,
      flags
    ),
    td\_yield\_all(td, chunkoff)
  );
  curval->used += chunkdatasize;
  if (curval->used == curval->value.size && vec->offset) {
    --(vec->offset);
  }
  return 0;
}

static
int td\_store\_chunk\_copy
  (
    td\_t*           td,
    unsigned        chunkoff,
    unsigned        chunkdatasize,
    unsigned        nextchunk,
    struct tdvvec*  vec,
    unsigned        refcount,
    unsigned        flags
  )
{
  struct tdvbuf* curval = &(vec->list[ vec->offset ]);
  unsigned char chunk[ chunkdatasize ];

  while (chunkdatasize >= curval->value.size - curval->used) {
    chunkdatasize -= (curval->value.size - curval->used);
    memcpy(
      &(chunk[ chunkdatasize ]),
      curval->value.data + curval->used,
      curval->value.size - curval->used
    );
    curval->used = curval->value.size;
    if (vec->offset == 0) {
      if (chunkdatasize) {
        return TDERR\_STRUCT;
      }
      break;
    }
    --(vec->offset);
    curval = &(vec->list[ vec->offset ]);
  }
  if (chunkdatasize) {
    memcpy(
      chunk,
      curval->value.data + curval->value.size - chunkdatasize,
      chunkdatasize
    );
    curval->used += chunkdatasize;
  }
  CHECK2(
    td\_store\_valuechunk(
      td,
      chunkoff,
      nextchunk,
      chunk,
      sizeof(chunk),
      refcount,
      flags
    ),
    td\_yield\_all(td, chunkoff)
  );
  return 0;
}

static
int td\_store\_chunk
  (
    td\_t*           td,
    unsigned        chunkoff,
    unsigned        chunkdatasize,
    unsigned        nextchunk,
    struct tdvvec*  vec,
    unsigned        refcount,
    unsigned        flags
  )
{
  struct tdvbuf* curval = &(vec->list[ vec->offset ]);

  if (chunkdatasize <= curval->value.size - curval->used) {
    CHECK(
      td\_store\_chunk\_nocopy(
        td, chunkoff, chunkdatasize, nextchunk, vec, refcount, flags
      )
    );
  } else {
    CHECK(
      td\_store\_chunk\_copy(
        td, chunkoff, chunkdatasize, nextchunk, vec, refcount, flags
      )
    );
  }
  return 0;
}

/**
 ingroup btree\_private

 Stores a new value in empty space(s). Returns its offset.

 param td Non-NULL pointer to an initialized btree structure.
 param[in] value Value container of data to be stored.
 param[in] refcount Reference count for this value node (min. 1).
 param[out] off On successful return, contains the offset within
 the btree's resource, of the first chunk.
 param[in] flags Bits from TDFLG\_* values.

 returns Zero on success, or a TDERR\_* value on error.
 

\subsubsection{Parameters}
\subsubsection{Returns}
\subsubsection{Called by}
