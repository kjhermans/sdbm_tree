
\newpage
\subsection{tdc\_mov}
\subsubsection{Declaration} Function prototype:

\begin{verbatim}
extern
int tdc_mov
  (tdc_t* tdc, const tdt_t* key, unsigned flags);
\end{verbatim}

\subsubsection{Description}


 ingroup btree

 Moves the cursor to a point at or near a given key.

 There are two modes of calling this function:
 - When a key is given, moves the cursor to the place AT, or JUST BEFORE the
   key, depending on the bits set in the 'flags' parameter.  Also renews the
   System Change Number cache in the cursor, renewing the cursor if it had
   become stale.
 - If a key isn't given, one the TDFLG\_BEGIN or TDFLG\_END bits must be
   set inside 'flags' parameter.  The cursor will then be set at the
   beginning or end of the btree, respectively.

 param tdc Pointer to an initialized tdc\_t structure, potentially stale.
 param key Pointer to an initialized dbt\_t structure, containing a key.
 param flags Flags; OR-ed together TDFLG\_PARTIAL or TDFLG\_EXACT, or zero,
 when a key is given, or one of TDFLG\_BEGIN or TDFLG\_END when it is not.

 returns Zero on success, TDERR\_NOTFOUND if the key cannot be found exactly
 and the TDFLG\_EXACT flag is set, or any of the errors of the underlying
 functions.

 Bear in mind that a partial key match is considered an
 exact match when TDFLG\_PARTIAL has been set during initialization.
 par Flags and matches:
 Since the terminology of 'partial' and 'exact' matches may be confusing,
 let's try to clear it up a bit more.

 par
 A partial match is a match using a search key that can be considered
 to be a part of the tested key; under normal circumstances
 (that is, without a customized 'compare' callback function) this means
 that the search key, up to its own size, contains the same bytes as the
 tested key (although the tested key may be longer).  A partial match is
 then passed from the search algorithm to the function-specific algorithm
 (get, put, delete, move etc.) as an exact match.  That function, in its
 turn, decides on whether it needs an exact match or not, and proceeds to
 run with it.  In the case of tdc\_mov(), this is left up to the caller.
 The following paragraph provides a diagram of this behaviour.

 par
 Given the list [ 'aaa', 'bbb', 'ccc' ], and the search keys
 'bbb' (perfect match), 'bb' (partial match) and 'aab' (no match),
 the cursor is moved to the following locations, given all TDFLG\_EXACT
 and TDFLG\_PARTIAL flag combinations:
 - Search 'bbb' with no flags; cursor at 'bbb'
 - Search 'bbb' with TDFLG\_PARTIAL; cursor at 'bbb'
 - Search 'bbb' with TDFLG\_PARTIAL|TDFLG\_EXACT; cursor at 'bbb'
 - Search 'bbb' with TDFLG\_EXACT; cursor at 'bbb'
 - Search 'bb' with no flags; cursor at 'aaa'
 - Search 'bb' with TDFLG\_PARTIAL; cursor at 'bbb'
 - Search 'bb' with TDFLG\_PARTIAL|TDFLG\_EXACT; cursor at 'bbb'
 - Search 'bb' with TDFLG\_EXACT; returns TDERR\_NOTFOUND
 - Search 'aab' with no flags; cursor at 'aaa'
 - Search 'aab' with TDFLG\_PARTIAL; cursor at 'aaa'
 - Search 'aab' with TDFLG\_PARTIAL|TDFLG\_EXACT; returns TDERR\_NOTFOUND
 - Search 'aab' with TDFLG\_EXACT; returns TDERR\_NOTFOUND

 par Using cursors in code:
 Unless you have only specified the TDFLG\_EXACT flag (and not the
 TDFLG\_PARTIAL flag, and verified the return value of this
 function to be zero, of course), you should always check where you are
 by using on of the key-filling cursor related functions, such as
 tdc\_get(), tdc\_nxt() or tdc\_itr().
 

\subsubsection{Parameters}
\subsubsection{Returns}
\subsubsection{Called by}
