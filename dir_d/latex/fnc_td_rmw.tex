
\newpage
\subsection{td\_rmw}
\subsubsection{Declaration} Function prototype:

\begin{verbatim}
extern
int td_rmw
  (
    td_t* td,
    const tdt_t* key,
    tdt_t* value,
    unsigned flags,
    rmwfnc_t callback,
    void* arg
  );
\end{verbatim}

\subsubsection{Description}


 ingroup btree

 Implements read-modify-write ('RMW') for this btree API.

 The idea behind RMW is that both node-examination and -manipulation
 can take place inside the same lock, allowing for certain data
 transactions to happen.
 The function caller provides a callback function as argument,
 which gets called as soon as the btree is locked and the key is found.
 Inside the * callback, the caller implements changes to the value tdt
 and returns zero. The renewed value is then stored and the lock
 is cleared.

 param td Non-NULL pointer to initialized td\_t structure.
 param key Non-NULL pointer to initialized tdt\_t structure.
 param value Non-NULL pointer to initialized tdt\_t structure.
 param flags Bits from TDFLG\_* values.
 param callback The callback that examines the value, potentially
 changes it, and returns zero to have it stored. Should any such function
 return TDERR\_NOTFOUND, then the value is considered unchanged, and
 zero is returned by the td\_rmw() function. Any other error is passed
 along to the caller of td\_rmw().
 param arg Any pointer, also NULL, that the caller wants passed to
 the callback.

 returns Zero on success, or non-zero on error.
 

\subsubsection{Parameters}
\subsubsection{Returns}
\subsubsection{Called by}
